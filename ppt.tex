\documentclass[12pt]{article}
\setlength{\textwidth}{6.5in}
\setlength{\textheight}{9.9in}
\setlength{\oddsidemargin}{0.0in}
\setlength{\evensidemargin}{0.0in}
\setlength{\topmargin}{-2.5 cm}
\setlength{\parskip}{0.7\baselineskip} % space between paragraphs
\setlength{\parindent}{0 cm} % increase this if you like paragraphs to be indented
\usepackage{graphicx}
\pagestyle{empty}

\newcommand{\e}{\mathrm{e}}

\title{ Praxis 1 revision}
\begin{document}
\maketitle

\begin{itemize}

\item Authority: professor, you believe what he/she says
\item authority: defer to someone knows more than me (classmates), still need to judge whether what they say is useful or helpful.
\item Frame: Defining an engineering opportunity with stakeholders and requirements (which backpack do i want? what makes a good backpack for me?)
\item Diverge: Generating ideas and exploring alternatives
\item Converge: Making decisions and justify recommendations
\item Engineers recommend designs
\item Engineers make decisions

\end{itemize}

\section{Toulmin's Structure of Argument}

\paragraph{Argument by example}

vulnerable to counter-examples

ground -- qualifier (for a reasonable artist) -- claim (this is a great pen) --(because) justification (The pen enables fine lines while still making strong blacks for contrast) -- (as shown by) evidence(a graph)

\begin{itemize}

\item is this alternative viable?  \textbf{constraints}
\item how does this alternative measure? \textbf{metrics}
\item how to these alternatives compare? \textbf{criteria}
\item all of which are based on \textbf{objectives}
\item \textbf{stakeholder}: \dots impacts / is impacted by \dots (eg. designer/design team)
\item objective: should do / be (Frame)
\item metrics: assess the design to enable evaluation (characteristic + unit: volume + litre) (Converge)
\item constraints: should not / must not (Converge)
\item criteria: want more / less (Converge)
\item high level objectives -- detailed objectives -- metrics -- criteria and/or constraints

\section{design for user-friendliness}

\paragraph{ground}

\paragraph{qualifier}

\paragraph{claim}

\paragraph{justification}

\paragraph{evidence}

\section{design for safety}

\paragraph{ground}

the existence of UL Standard for Safety for Electric Fans

\paragraph{qualifier}

Without breaking the external structure

\paragraph{claim}

the portable fan is safe

\paragraph{justification}



\paragraph{evidence}



\section{design for the environment}
\paragraph{objective}

 design for the environment
 
\paragraph{metrics} 

number of non-rechargeable AA batteries disposed per year. unit: number of items.
the rechargeable battery life. unit: hour

\paragraph{criteria}

the fewer non-rechargeable AA batteries disposed per year, the more friendly to the environment.
the longer rechargeable battery life, the more friendly to the environment.

\paragraph{constraints}

the e.m.f. of the rechargeable battery must not be lower than 5v

\item secondary research: Finding sources for proof, information, processes, ‘reference designs’, and related concepts.

\item primary research:  Testing through experimentation, modeling, surveying or prototyping. Discovering something new.

\item CRAAP test: Currency, Relevance, Authority, Accuracy, Purpose 


\end{itemize}

\begin{figure}[h!]
\begin{center}
\includegraphics[width=\textwidth]{claim}
\caption{How could I support the claim that the designers intended this dishwasher to be sustainable}
\end{center}
\end{figure}

\begin{figure}[h!]
\begin{center}
\includegraphics[width=\textwidth]{requirement}
\caption{How would I write a requirement related to the energy efficiency regulations}
\end{center}
\end{figure}

\begin{itemize}

\item  ground--analytical claim--interpretive claim--speculative claim--crazy idea

as I add evidence, I reduce the burden for interpretation.

\end{itemize}

\end{document}